%% -------------------------------------------------------------------------
%% Created by Joseph Dorn, School of Physics & Astronomy, Cardiff University
%% This report is part of my personal project
%% -------------------------------------------------------------------------

\documentclass[11pt,twoside]{article}

% LaTeX packages & global styles
% Packages used
\usepackage{geometry}
% Fonts
\usepackage[T1]{fontenc}
\usepackage{mathpazo}% math
\usepackage{utopia}% text serif
\usepackage{avant}% text sans serif
% \usepackage[math]{fourier}
% \mathversion{fourier}
% \usepackage{mathpazo}
% \usepackage{linguisticspro}
% \usepackage{kpfonts}
% \usepackage{ClearSans}
% \usepackage{montserrat}
% \usepackage{inter}
%
\usepackage{amsmath}
\usepackage{units}
%
\usepackage{multicol}
%
\usepackage{graphicx}
%
\usepackage{tabularx}
%
% http://mirror.ox.ac.uk/sites/ctan.org/macros/latex/contrib/biblatex/doc/biblatex.pdf
\usepackage[citestyle=authoryear-comp,bibstyle=authoryear,sorting=none,url=false]{biblatex}
\addbibresource{biblio.bib}
%
\usepackage{titling}% https://ctan.org/pkg/titling
%
\usepackage{titlesec}
%
\usepackage{sectsty}
% adds a dot after section numbering
\usepackage{secdot}
%
\usepackage{changepage}% http://ctan.org/pkg/changepage
%
\usepackage{lipsum}% https://ctan.org/pkg/lipsum
% adds hyperlinks
\usepackage[bookmarks,breaklinks,colorlinks=true,allcolors=blue]{hyperref}% https://ctan.org/pkg/hyperref
% 

% Headers & Footers %% https://tex.stackexchange.com/questions/243378/how-do-i-create-a-header-that-has-a-colored-box-with-page-number-chaptername-a
\usepackage{fancyhdr}
\usepackage[dvipsnames,usenames]{color}
%

%


%% Font graveyard
% \usepackage[rmdefault]{stix2}
% \usepackage[T1]{fontenc}
% \usepackage[T1]{fontenc}
% \usepackage{tgbonum}

% \usepackage{ebgaramond}
% \usepackage[default]{droidserif}

% \usepackage{Baskervaldx}
% \usepackage{sourceserifpro}
% \usepackage{charter}
% Sans Serif
% \usepackage{opensans}
% Margins 
\geometry{
    a4paper,
    %width=197mm,
    right=6.5mm,
    left=6.5mm,
    top=16mm,
    bottom=18mm,
    % showframe=true
    % margin=0.3in,
    footskip=-0.25in
}


% [2]Column Separation 
\setlength{\columnsep}{1.5em}


% Graphics Path
\graphicspath{ {./graphics/} }


% Fonts
\sectionfont{\LARGE}
\subsectionfont{\Large}%\underline}
\subsubsectionfont{\slshape}


% Headers/footers
\setlength\headheight{19.6pt}
\pagestyle{fancy}

\renewcommand{\sectionmark}[1]{ \markright{#1}{} }
\renewcommand{\subsectionmark}[1]{}

\fancyhf{}

\fancyhead[LE]{%
  \makebox[40pt][r]{
    \colorbox{green}{\makebox[\textwidth][r]{\sffamily\large\bfseries\textcolor{black}{\thepage}\enskip}}\hspace*{1em}}%
  {\sffamily\large\itshape\rightmark\/}\hspace*{1em}\headrulefill%
}

\fancyhead[RO]{%
  \mbox{}\headrulefill \hspace*{1em}{\sffamily\large\itshape\rightmark}%
  \makebox[40pt][l]{%
    \hspace*{1em}\colorbox{green}{\makebox[\textwidth][l]{\enskip\sffamily\large\bfseries\textcolor{black}{\thepage}}}}%
}

\fancyfoot[LO,RE]{\sffamily\itshape\ Joseph Dorn\/}
\fancyfoot[RO,LE]{\sffamily {\slshape N}--body modelling of young stellar systems}

\renewcommand\headrulewidth{0pt}

\def\headrulefill{\leaders\hrule width 0pt height 3pt depth -2.8pt \hfill}

\renewcommand{\footrulewidth}{0.6pt}


%% ------------------------------------------------------
%% ETC.
% \usepackage{opensans}
% \sectionfont{\usefont{T1}{pag}{b}{n}\selectfont\LARGE}
% \sectionfont{\opensans\selectfont\LARGE}
% \subsectionfont{\usefont{T1}{pag}{b}{n}\selectfont\large\underline}
% \subsectionfont{\opensans\selectfont\large\underline}


% \fancyhead{}
% \fancyhead[RO]{N-body modelling of young stellar systems\quad\Large\thepage}
% % \fancyhead[RO,LE]{\thesection}
% \fancyhead[LO,RE]{\leftmark}

% \fancyfoot{}
% \fancyfoot[C]{\leftmark}
% \fancyfoot[RO]{Joseph Dorn}

% \fancyfoot[CO]{Interim Report}
% \fancyfoot[CO]{Section \thesection}
% \fancyfoot[Lo,RE]{\itshape\ Interim Report\/}

% \renewcommand{\headrulewidth}{0.1pt}
% \renewcommand{\footrulewidth}{0.4pt}


% Beginning of document
\begin{document}


% Cover page, title page [including un-numbered]
\begin{center}


% CU logo
\begin{figure}
    \centering
    \vspace*{4em}
    \includegraphics[width=3.5cm]{cu_logo.jpg}
\end{figure}

\vspace*{2em}

% Title
{\huge\scshape\bfseries School of Physics and Astronomy}\\
\vspace*{2em}
{\Huge\scshape\bfseries Year 3 Interim Project Report}\\
\vspace*{1em}
{\LARGE\scshape Session 2022--2023}\\

\vspace*{4em}

% Student info
\begin{tabularx}{0.8\textwidth} { 
    | >{\raggedright\arraybackslash\bfseries}l 
      >{\raggedleft\arraybackslash}X }
    Name & Joseph Henry Dorn \\ [0.5ex]
    Student Number  & C1940888 \\ [0.5ex]
    Degree Programme  & BSc Astrophysics (UFBSASTA) \\ [0.5ex]
    Project Title  & N-body Modelling of Young Stellar Systems \\ [0.5ex]
    Supervisor  & Dr.\ Paul Clark \\ [0.5ex]
    Assessor  & Dr.\ Mikako Matsuura \\ [0.5ex]
\end{tabularx}

\vspace*{8em}

% Declaration
\begin{adjustwidth}{0.1\textwidth}{0.1\textwidth}
    {\large\bfseries Declaration} \\ [0ex]
    I have read and understand Appendix 2 in the Student Handbook: “Some advice on the avoidance of plagiarism”. \\ [2ex] 
    I hereby declare that the attached report is exclusively my own work, that no part of the work has previously been submitted for assessment (although do note that material in “Interim Report” may be re-used in the final “Project Report” as it is considered part of the same assessment), and that I have not knowingly allowed it to be copied by another person.
\end{adjustwidth}

\end{center}

\thispagestyle{empty} % Prevents page number from being included on the cover
% \clearpage\setcounter{page}{1} % Start including page numbers from here
% \pagenumbering{roman} % in roman numerals

\title{\vspace*{-1em}\huge\textbf{
    {\slshape N}-body modelling of young stellar systems
    }\vspace{-0.7em}}

\author{\bfseries Joseph Dorn\\ [0.5ex]
    \small\slshape under the direction of\\
    \small Dr.\ Paul Clark
    \vspace{0.7em}\\
    {\small Cardiff University, School of Physics and Astronomy\/}
    \vspace{-2em}}

\date{ }

\maketitle\vspace*{-2em}
% Abstract
\noindent
\hrulefill{}
\vspace*{-1.5em}
\section*{\centering\scshape\large Abstract}
\vspace*{-1em}
\noindent
\lipsum[12] % text will go here...

\noindent
\hrulefill{}

\vspace*{\fill}

% TOC of document
\begingroup
    \hypersetup{linkcolor=black}
    \tableofcontents
\endgroup

\thispagestyle{empty} % Prevents page number from being included on the cover
\clearpage\setcounter{page}{1} % Start including page numbers from here
\pagenumbering{arabic} % 

% \addcontentsline{toc}{section}{Abstract} 


%%% --------------------------------------------

\section{Introduction}
In 1617, the first observation of [visual] binary stars was made by Galileo Galilei; he discovered that the second star from the end of the Big Dipper constellation' handle was actually comprised of two stars; later this was revised to six stars. However, it wasn't until shortly after the birth of modern astronomy in the 17th century that Sir William Herschell observed and catalogued $\sim$700 pairs of stars, first coining the term `binary' when referencing these observations. The importance of these peculiar stellar systems was first realised by~\cite{kuiperProblemsDoubleStarAstronomy1935}, who suggested that the physical processes involved throughout the evolution of stellar populations could be theorised if we can determine the distribution of key orbital parameters and the muliplicity frequency of binary systems. 

Whilst the past few decades have brought instrumentation breakthroughs that have enabled extensive observational research into binaries and multiple systems, the technological advancements that allow computationally intensive {\slshape N\/}-body simulations of the Universe to be run have allowed theoretical and observational astrophysics to be extensively tested programmatically and compared to what is observed, allowing for a very interdisciplinary field of researchers to rapidly further progress. 

In this paper we will attempt to model the early phases of the stars in young stellar systems to see how quickly these stars are ejected from their protostellar core. We will also attempt to model the properties of the binary and triple-star systems that form by dynamical capture during this phase of the clusters stellar evolution. This will be achieved by constructing an {\slshape N\/}-body model simulation.


% were upon turning his telescope toward the second star from the end of the handle of the Big Dipper constellation, Galileo Galilei discovered that what he thought was a single star was actually two stars; eventually it was found to comprise of six stars. This was the first time visual binaries were observed. 







% Binary stellar systems account for formation and evolution While the physical processes involved in the formation and evolution of single stars and binaries are generally agreed upon {\parencites[see][]{toonenPopCORNHuntingDifferences2014}{postnovEvolutionCompactBinary2014}}, the processes governing multiple/exotic systems are poorly understood {\parencite[see][]{toonenEvolutionHierarchicalTriple2016}}. We know that

% \hrule

\begin{multicols}{2}

    \section{Background}

{ \lipsum[14] }

\subsection{Binary Properties}

{ \lipsum[1-2] }

\subsection{Star Formation}

{ \lipsum[8-9] }

\subsubsection{Time Scales}

{ \lipsum[15-16] }

\subsection{Binary Formation}

{ \lipsum[6-7] }

\subsection{{\slshape N}-body Problem}

{ \lipsum[1] }
    % Suppress hbox warnings
\hbadness=99999  % or any number >=10000

\section{Methodology}
{\slshape N\/}-body simulations enable the evolution of a system of continuously interacting bodies to be numerically approximated. Here, we describe our astrophysical simulation implementation of this, where each body represents an individual young star contained in a small-{\slshape N\/} cluster, and these bodies gravitationally interact with every other body. The simulation will contain multiple small-{\slshape N\/} clusters separated by radius $2R$, where $R$ is the radius of each individual cluster, and $R<10^{4}AU$.

\subsection{{\slshape N\/}-body Algorithm}
Multiple algorithms exist for {\slshape N\/}-body simulations. Hierarchical {\slshape N\/}-body algorithms, such as treecodes~\parencite{barnesHierarchicalLogForcecalculation1986}, fast multipole methods (FMM)~\parencite{lfFastAlgorithmParticle2001}, and hybrid treecode/FMM algorithms {\parencites{dehnenHierarchicalForceCalculation2002}{chengFastAdaptiveMultipole1999a}}, greatly reduce the computational complexity of the simulation by approximating some of the body-body interactions. The treecode algorithm approximates long-range forces by replacing groups of remote particles with their centre of mass, bringing the complexity down to $O(N \log N)$, while FMMs additionally group nearby particles, further reducing the complexity to just $O(N)$.

Whilst these hierarchical methods significantly speed up calculations, they do introduce a small amount of error. Their application is perfectly suited for large-scale {\slshape N\/}-body simulations, where {\slshape N\/} is large and the simulation is of structures orders of magnitudes more massive and complex than the small-{\slshape N\/} clusters being studied in this paper, as at that scale the errors introduced by the afformentioned hierarchical algorithms become negligible. Additionally, the direct simulation of an {\slshape N\/}-body problem using the {\slshape all-pairs\/} approach, where all body-body forces are computed, has a significant computational complexity of $O(N^2)$; for large {\slshape N\/} this is simply too expensive. However, given the scientific objective of this paper, the brute-force {\slshape all-pairs\/} approach will be used despite it's computational complexity of order $N^2$, as we are simulating small-{\slshape N\/} clusters and need to minimise errors.

\subsection{All-Pairs Approach}

In the following equations, we signify vectors (generally in 3D) using bold font. For this simulation we use Newtonian equations of gravitational force. The most basic form of this is given by the following:
\[ 
    F=\frac{Gm_{1}m_{2}}{r^2}, 
\]
where $F$ is the magnitude of the force acting between the two bodies; $m_{1}$ and $m_{2}$ are the masses of the two objects; $r$ is the distance from the centre of mass for each body; and $G$ is the gravitational constant. 

In order to implement this in our {\slshape N\/}-body simulation, we must additionally calculate the direction of the force. Given {\slshape N\/} bodies with position and velocity $\mathbf{x}_i$ and $\mathbf{v}_i$ respectively, where $1\leq i\leq N$, the resulting force vector $\mathbf{f}_{ij}$ acting upon body $i$ caused by its gravitational interaction with body $j$ is given by:
\[
    \mathbf{f}_{ij}=\underbrace{G\frac{m_{i}m_{j}}{\|\mathbf{r}_{ij}\|^2}}_{magnitude} \cdot \underbrace{\frac{\mathbf{r}_{ij}}{\|\mathbf{r}_{ij}\|}}_{direction},
\]
where $m_{i}$ and $m_{j}$ are the masses of bodies $i$ and $j$, respectively; $\mathbf{r_{ij}}$ is the vector from the centre of body $i$ to body $j$ where $\mathbf{r_{ij}}=\mathbf{x}_i - \mathbf{x}_j$; and $G$ is the gravitational constant. The {\itshape magnitude\/} of the force is proportional to the product of the two bodies masses and is inversely proportional to the square of the distance between body $i$ and body $j$. Given that gravitational forces are attractive, the {\itshape direction\/} of the force is given by the unit vector going from body $i$ to body $j$.

In order to obtain the total force acting on body $i$, $\mathbf{F}_i$, every interaction that body $i$ has with all other $N-1$ bodies is summed:
\[
    \mathbf{F}_i=\sum\limits_{\substack{1\leq j\leq N \\ j\neq i }}\mathbf{f}_{ij}=Gm_{i}\cdot \sum\limits_{\substack{1\leq j\leq N \\ j\neq i }}\frac{m_{j}\mathbf{r}_{ij}}{\|\mathbf{r}_{ij}\|^3}.
\]

Newtonian equations of gravitational force only provide an approximation of the effects of gravity as both bodies are treated as being point-masses; the bodies size is not accounted for. When bodies approach each other, the resultant force, $\mathbf{F}_{i}$, grows without bounds towards infinity. This presents an issue for both the numerical integration required in this simulation and for the physical accuracy of this study. Typically, astrophysical simulations presume a collisionless interaction between bodies where it is appropriate and where collisions are not being studied. We therefore introduce a {\itshape softening factor\/}, $\epsilon^2>0$; this is further explained in \autoref{sec:softening}, along with the value we use for $\epsilon$. The equation is rewritten as:
\[
    \mathbf{F}_i\approx Gm_{i}\cdot \sum_{1\leq j\leq N}\frac{m_{j}\mathbf{r}_{ij}}{(\|\mathbf{r}_{ij}\|^2+\epsilon^2)^{\nicefrac{3}{2}}}.
\]
Note that when $\epsilon^2>0$, $\mathbf{f}_{ij}=0$, so the condition $j\neq i$ is no longer required. To integrate the body-body interactions over time and update the position and velocity of body $i$, the acceleration $\mathbf{a}_{i}=\mathbf{F}_{i}/m_{i}$ must be calculated. We can therefore simplify the equation to:
\[ 
    \mathbf{a}_i\approx G\cdot \sum_{1\leq j\leq N}\frac{m_{j}\mathbf{r}_{ij}}{(\|\mathbf{r}_{ij}\|^2+\epsilon^2)^{\nicefrac{3}{2}}}.
\]

\subsubsection{Softening Factor}\label{sec:softening}

{ \lipsum[9] }

\subsection{Integrator Scheme}

{ \lipsum[2-4] }

\subsection{Initial Conditions}

{ \lipsum[10] }


\end{multicols}

\newpage

\begin{multicols*}{2}
    \printbibliography\
\end{multicols*}

\end{document}

% #### Intro
% - Mention the **goals** of the project, making reference to them with where the current state of the project, analysis, and results are [0]
% - **N-body problem** [5/4]
% 	- Different methods/integrators
% - **Core "system" dissolution [4/5]**
% - **Binary formation [3]** - *Matthew Bate (Exeter)*
% 	- Capture ->
% 		- 3-body encounters
% 		- Dissipatin gas encounters
% 	- Fragmentation
% - **Binary properties [1]**
% 	- Why are binary stars important?
% 	- What do we know and what are we yet to find out about them?
% - **Star formation [2] **
% 	- Molecular cloud collapse
% 	- Jeans Mass
% 	- Prestellar cores
% 	- Class 0/I/II/III
% 		- Assuming we are at class II/III,
% 		- Mention class III also has a disk around it, which we will not be using.
% 	- Where star formation takes place in molecular clouds
% 		- Supersonic turbulence,
% 		- Structure/filaments.

% The **N-body problem** and **core "system" dissolution** are linked together, talk about:
% - What has been done and previously studied?
% - *What are we going to do?*
% 	- Type of n-body
% 	- What is being studied? (setup)
% 	- What we will report and why





% \begin{multicols}{2}
% [
% \section{Section One}
% This is the first section and we are placing some text here.
% ]
% Now, we are adding some text here to understand how multiple columns will come up.
% This way, we hope that you are understanding the multicols package and its usage.
% The text that we have placed here is only for sample.
% \columnbreak
% This is again some new text here. Just to show you how the..
% text is getting placed…..
% Hope
% you are able to understand…
% \end{multicols}
% Some more text here just to show.
