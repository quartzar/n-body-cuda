\section{Introduction}
In 1617, the first observation of [visual] binary stars was made by Galileo Galilei; he discovered that the second star from the end of the Big Dipper constellation' handle was actually comprised of two stars; later this was revised to six stars. However, it wasn't until shortly after the birth of modern astronomy in the 17th century that Sir William Herschell observed and catalogued $\sim$700 pairs of stars, first coining the term `binary' when referencing these observations. The importance of these peculiar stellar systems was first realised by~\cite{kuiperProblemsDoubleStarAstronomy1935}, who suggested that the physical processes involved throughout the evolution of stellar populations could be theorised if we can determine the distribution of key orbital parameters and the muliplicity frequency of binary systems. 

Whilst the past few decades have brought instrumentation breakthroughs that have enabled extensive observational research into binaries and multiple systems, the technological advancements that allow computationally intensive {\slshape N\/}-body simulations of the Universe to be run have allowed theoretical and observational astrophysics to be extensively tested programmatically and compared to what is observed, allowing for a very interdisciplinary field of researchers to rapidly further progress. 

In this paper we will attempt to model the early phases of the stars in young stellar systems to see how quickly these stars are ejected from their protostellar core. We will also attempt to model the properties of the binary and triple-star systems that form by dynamical capture during this phase of the clusters stellar evolution. This will be achieved by constructing an {\slshape N\/}-body model simulation.


% were upon turning his telescope toward the second star from the end of the handle of the Big Dipper constellation, Galileo Galilei discovered that what he thought was a single star was actually two stars; eventually it was found to comprise of six stars. This was the first time visual binaries were observed. 







% Binary stellar systems account for formation and evolution While the physical processes involved in the formation and evolution of single stars and binaries are generally agreed upon {\parencites[see][]{toonenPopCORNHuntingDifferences2014}{postnovEvolutionCompactBinary2014}}, the processes governing multiple/exotic systems are poorly understood {\parencite[see][]{toonenEvolutionHierarchicalTriple2016}}. We know that
